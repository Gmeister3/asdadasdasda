
% \documentclass[review=false, screen=true]{acmart}
\documentclass[format=acmsmall, review=false, screen=true]{acmart}

% Set letter paper size:
% \setlength{\paperheight}{11in}
% \setlength{\paperwidth}{8.5in}
% \usepackage[
%   % pass,% keep layout unchanged 
%   % showframe,% show the layout
%   % a4paper
% ]{geometry}


% Package to generate and customize Algorithm as per ACM style
\usepackage[linesnumbered,ruled,vlined]{algorithm2e}
% \usepackage[ruled]{algorithm2e}
\renewcommand{\algorithmcfname}{ALGORITHM}
% \SetAlFnt{\small}
% \SetAlCapFnt{\small}
% \SetAlCapNameFnt{\small}
% \SetAlCapHSkip{0pt}
% \IncMargin{-\parindent}



\usepackage{booktabs}
\usepackage{tabularx}  

\usepackage{amsmath,amssymb,amsfonts}

\usepackage{enumitem}

\usepackage{lipsum} 

\SetKwRepeat{Do}{do}{while} 


\usepackage{lmodern}
\usepackage{courier}

\usepackage{listings}
\lstset{
  language={Pascal},
  backgroundcolor=\color{white},   % choose the background color; you must add \usepackage{color} or \usepackage{xcolor}; should come as last argument
  basicstyle=\ttfamily\small,      %\footnotesize,        % the size of the fonts that are used for the code
  breakatwhitespace=false,         % sets if automatic breaks should only happen at whitespace
  breaklines=true,                 % sets automatic line breaking
  captionpos=b,                    % sets the caption-position to bottom
  commentstyle=\color{red},        % comment style
  frame=single,                    % adds a frame around the code
  keepspaces=true,                 % keeps spaces in text, useful for keeping indentation of code (possibly needs columns=flexible)
  keywordstyle=\color{blue},       % keyword style
  language=Octave,                 % the language of the code
  showspaces=false,                % show spaces everywhere adding particular underscores; it overrides 'showstringspaces'
  showstringspaces=false,          % underline spaces within strings only
  showtabs=false,                  % show tabs within strings adding particular underscores
  stringstyle=\color{brown},       % string literal style
  tabsize=2,                       % sets default tabsize to 2 spaces
}

\usepackage{setspace}




% Document starts
\begin{document} 
% Title portion. Note the short title for running heads 
\title[COSC 3P91 -- Assignment 1]{COSC 3P91 -- Assignment 1 -- Question 2}  

\author{Design Description Document}
\affiliation{%
  \institution{Brock University}
  \streetaddress{1812 Sir Isaac Brock Way}
  \city{St. Catharines}
  \state{ON}
  \postcode{L2S 3A1}
  \country{Canada}
}

\begin{abstract}

This document presents an object-oriented design for a strategic territory conquest simulation system. The architecture employs abstraction hierarchies, polymorphic behaviors, and modular packaging to create an extensible framework. Seven distinct modules organize functionality: territorycore for domain management, fortifications for defensive capabilities, dwellers for population entities, assaultunits for offensive forces, gamecontrol for orchestration logic, playerspace for participant data, and frontend for interaction mechanisms.

\end{abstract}







\maketitle


\section{Network Security Questions}

Add your assignment answers here... Include tables, algorithms, and figures if needed...

\begin{figure}[h!]
  \centering
  \includegraphics[width=0.3\linewidth]{images/rosette.eps} 
  \caption{Example of an image. (you can construct more sophisticated layouts using subfigs)}
\end{figure}

\begin{table}[h]
  \centering
  % \small
  \caption{Example of a table (nicely looking tables with booktabs).}
  \begin{tabular}{llll} %p{1.7cm}
    \toprule
    \textbf{Name} & \textbf{ID} & \textbf{Subtotal} & \textbf{Total} \\  
    \midrule
  John Doe & jd2001 & 75  & 89 \\
    \bottomrule
  \end{tabular}
\end{table}  

\begin{algorithm}[t]
  \footnotesize
  \caption{Example of algorithm}
  \label{algo1}
  \KwData{$X_i$; $P_{a_i}$; $R$; $A_i$; $\gamma$; $\epsilon$ }
  \KwResult{$\pi_i$; $v^*$ } 

  $V=0$; $\pi_i=0$\;
  \Do{$\Delta < \epsilon$}{
    $\Delta = 0$\;
    \For{$x \in X$}{
      $Av=0$\;
      \For{$a \in A$}{
        $x_{n} = A(x)$\;
        $Av[a] = P_a[x][x_{n}]*(R[x_{n}]+\gamma*V[x_{n}])$\;
      }
      $av_{best} = max(Av)$\;
      $\Delta = max(\Delta, |av_{best}-V[x]|)$\;
      $V[x] = av_{best}$\;
      $\pi_i[x]=argmax(Av)$\;
    }
  }
\end{algorithm}


\subsection{Question 1}

\lipsum[2-4]



\section{Secured Java Application}

The description of your Java code is here... Include any auxiliary text object for aiding your description (tables, algorithms, figures...)

Do not forget to cite sources~\cite{GnutellaNet2002} if any!


\subsection{Performance Analysis}

\lipsum[2-4]


\subsection{Conclusions}


\bibliographystyle{ACM-Reference-Format-Journals}
\bibliography{acmsmall-sample-bibfile}


\appendix

\section{Resources Package - ResourceType Enum}
\begin{spacing}{0.8}
\begin{lstlisting}[language=Java]
package resources;

public enum ResourceType {
    GOLD,
    IRON,
    WOOD,
    FOOD
}
\end{lstlisting}
\end{spacing}

\section{Resources Package - Cost Class}
\begin{spacing}{0.8}
\begin{lstlisting}[language=Java]
package resources;

public class Cost {
    private int gold;
    private int iron;
    private int wood;
    private int time;
    
    public Cost(int gold, int iron, int wood, int time) {
        this.gold = gold;
        this.iron = iron;
        this.wood = wood;
        this.time = time;
    }
    
    public int getTotalValue() {
        return gold + iron + wood;
    }
    
    public Cost multiply(double factor) {
        return new Cost(
            (int)(gold * factor),
            (int)(iron * factor),
            (int)(wood * factor),
            (int)(time * factor)
        );
    }
    
    public int getGold() { return gold; }
    public int getIron() { return iron; }
    public int getWood() { return wood; }
    public int getTime() { return time; }
}
\end{lstlisting}
\end{spacing}

\clearpage

\section{Resources Package - ResourceManager Class}
\begin{spacing}{0.8}
\begin{lstlisting}[language=Java]
package resources;

public class ResourceManager {
    private int gold;
    private int goldCapacity;
    private int iron;
    private int ironCapacity;
    private int wood;
    private int woodCapacity;
    private int foodProduction;
    private int foodConsumption;
    
    public void addGold(int amount) {
        gold = Math.min(gold + amount, goldCapacity);
    }
    
    public void addIron(int amount) {
        iron = Math.min(iron + amount, ironCapacity);
    }
    
    public void addWood(int amount) {
        wood = Math.min(wood + amount, woodCapacity);
    }
    
    public boolean consumeResources(Cost cost) {
        if (hasEnoughResources(cost)) {
            gold -= cost.getGold();
            iron -= cost.getIron();
            wood -= cost.getWood();
            return true;
        }
        return false;
    }
    
    public boolean hasEnoughResources(Cost cost) {
        return gold >= cost.getGold() && 
               iron >= cost.getIron() && 
               wood >= cost.getWood();
    }
    
    public Cost getAvailableLoot() {
        return new Cost(gold / 2, iron / 2, wood / 2, 0);
    }
    
    public boolean canSupport(int population) {
        return foodProduction >= foodConsumption + population;
    }
}
\end{lstlisting}
\end{spacing}

\clearpage

\section{Buildings Package - Building Abstract Class}
\begin{spacing}{0.8}
\begin{lstlisting}[language=Java]
package buildings;

import resources.Cost;

public abstract class Building {
    protected int level;
    protected int maxLevel;
    protected int hitPoints;
    protected int maxHitPoints;
    protected int positionX;
    protected int positionY;
    protected Cost buildCost;
    protected int buildTime;
    protected boolean isUpgrading;
    protected long upgradeStartTime;
    
    public Building(int maxLevel, int maxHitPoints, Cost buildCost) {
        this.maxLevel = maxLevel;
        this.maxHitPoints = maxHitPoints;
        this.hitPoints = maxHitPoints;
        this.buildCost = buildCost;
        this.level = 1;
        this.isUpgrading = false;
    }
    
    public void upgrade() {
        if (!isMaxLevel() && !isUpgrading) {
            isUpgrading = true;
            upgradeStartTime = System.currentTimeMillis();
        }
    }
    
    public Cost getUpgradeCost() {
        return buildCost.multiply(level * 1.5);
    }
    
    public boolean isMaxLevel() {
        return level >= maxLevel;
    }
    
    public void takeDamage(int damage) {
        hitPoints = Math.max(0, hitPoints - damage);
    }
    
    public void repair() {
        hitPoints = maxHitPoints;
    }
    
    public void completeUpgrade() {
        if (isUpgrading) {
            level++;
            isUpgrading = false;
            maxHitPoints = (int)(maxHitPoints * 1.2);
            hitPoints = maxHitPoints;
        }
    }
    
    public int getLevel() { return level; }
    public int getHitPoints() { return hitPoints; }
}
\end{lstlisting}
\end{spacing}

\clearpage

\section{Buildings Package - VillageHall Class}
\begin{spacing}{0.8}
\begin{lstlisting}[language=Java]
package buildings;

import resources.Cost;

public class VillageHall extends Building {
    private int villageLevel;
    
    public VillageHall() {
        super(10, 5000, new Cost(1000, 500, 500, 600));
        this.villageLevel = 1;
    }
    
    public int getAllowedBuildingLevel() {
        return villageLevel;
    }
    
    public int getVillageLevel() {
        return villageLevel;
    }
    
    @Override
    public void completeUpgrade() {
        super.completeUpgrade();
        villageLevel = level;
    }
}
\end{lstlisting}
\end{spacing}

\section{Buildings Package - Farm Class}
\begin{spacing}{0.8}
\begin{lstlisting}[language=Java]
package buildings;

import resources.Cost;

public class Farm extends ProductionBuilding {
    private int foodPerHour;
    private int populationSupport;
    
    public Farm() {
        super(8, 1000, new Cost(100, 50, 200, 120));
        this.foodPerHour = 50;
        this.populationSupport = 10;
        this.maxWorkers = 3;
        this.productionRate = 20;
    }
    
    public int getFoodProduction() {
        return foodPerHour + getProduction();
    }
    
    public int getPopulationSupport() {
        return populationSupport * level;
    }
}
\end{lstlisting}
\end{spacing}

\clearpage

\section{Buildings Package - ArcherTower Class}
\begin{spacing}{0.8}
\begin{lstlisting}[language=Java]
package buildings;

import resources.Cost;

public class ArcherTower extends DefenseBuilding {
    private int arrowCount;
    
    public ArcherTower() {
        super(8, 1500, new Cost(300, 200, 100, 240));
        this.damage = 50;
        this.range = 10;
        this.attackSpeed = 1.0;
        this.arrowCount = 100;
    }
    
    public void reload() {
        arrowCount = 100;
    }
}
\end{lstlisting}
\end{spacing}

\section{Inhabitants Package - Inhabitant Abstract Class}
\begin{spacing}{0.8}
\begin{lstlisting}[language=Java]
package inhabitants;

import resources.Cost;

public abstract class Inhabitant {
    protected int level;
    protected int maxLevel;
    protected int hitPoints;
    protected int maxHitPoints;
    protected Cost trainingCost;
    protected int trainingTime;
    protected int foodConsumption;
    protected boolean isUpgrading;
    
    public Inhabitant(int maxLevel, int maxHitPoints) {
        this.maxLevel = maxLevel;
        this.maxHitPoints = maxHitPoints;
        this.hitPoints = maxHitPoints;
        this.level = 1;
        this.isUpgrading = false;
    }
    
    public void upgrade() {
        if (!isUpgrading && level < maxLevel) {
            isUpgrading = true;
        }
    }
    
    public Cost getUpgradeCost() {
        return trainingCost.multiply(level * 1.5);
    }
    
    public void takeDamage(int damage) {
        hitPoints = Math.max(0, hitPoints - damage);
    }
    
    public boolean isAlive() {
        return hitPoints > 0;
    }
    
    public void heal() {
        hitPoints = maxHitPoints;
    }
}
\end{lstlisting}
\end{spacing}

\clearpage

\section{Inhabitants Package - Worker Class}
\begin{spacing}{0.8}
\begin{lstlisting}[language=Java]
package inhabitants;

import resources.Cost;
import buildings.Building;

public class Worker extends Inhabitant {
    private boolean isIdle;
    private String currentTask;
    
    public Worker() {
        super(5, 100);
        this.trainingCost = new Cost(50, 0, 0, 30);
        this.foodConsumption = 1;
        this.isIdle = true;
        this.currentTask = "";
    }
    
    public void buildStructure(Building b) {
        isIdle = false;
        currentTask = "Building";
    }
    
    public void produceFood() {
        isIdle = false;
        currentTask = "Producing Food";
    }
    
    public void repair(Building b) {
        isIdle = false;
        currentTask = "Repairing";
    }
    
    public void setIdle(boolean idle) {
        this.isIdle = idle;
        if (idle) {
            currentTask = "";
        }
    }
}
\end{lstlisting}
\end{spacing}

\section{Inhabitants Package - Soldier Class}
\begin{spacing}{0.8}
\begin{lstlisting}[language=Java]
package inhabitants;

import resources.Cost;

public class Soldier extends ArmyUnit {
    private int shield;
    
    public Soldier() {
        super(6, 150);
        this.trainingCost = new Cost(100, 50, 0, 60);
        this.damage = 30;
        this.attackRange = 1;
        this.movementSpeed = 2;
        this.shield = 20;
        this.foodConsumption = 2;
    }
    
    public void block() {
        // Implementation for blocking attacks
    }
}
\end{lstlisting}
\end{spacing}

\clearpage

\section{Core Package - Village Class}
\begin{spacing}{0.8}
\begin{lstlisting}[language=Java]
package core;

import java.util.List;
import java.util.ArrayList;
import buildings.Building;
import buildings.VillageHall;
import inhabitants.Inhabitant;
import resources.ResourceManager;

public class Village {
    private String name;
    private int level;
    private int population;
    private int maxPopulation;
    private int area;
    private int maxArea;
    private VillageHall villageHall;
    private List<Building> buildings;
    private List<Inhabitant> inhabitants;
    private ResourceManager resources;
    private int defenseScore;
    private int attackScore;
    private long guardPeriodEnd;
    private boolean inGuardPeriod;
    
    public Village(String name) {
        this.name = name;
        this.level = 1;
        this.population = 0;
        this.maxPopulation = 50;
        this.area = 0;
        this.maxArea = 100;
        this.buildings = new ArrayList<>();
        this.inhabitants = new ArrayList<>();
        this.resources = new ResourceManager();
        this.villageHall = new VillageHall();
        this.inGuardPeriod = true;
        this.guardPeriodEnd = System.currentTimeMillis() 
            + (7 * 24 * 60 * 60 * 1000);
    }
    
    public boolean addBuilding(Building b) {
        if (canBuild(b)) {
            buildings.add(b);
            return true;
        }
        return false;
    }
    
    public void removeBuilding(Building b) {
        buildings.remove(b);
    }
    
    public boolean addInhabitant(Inhabitant i) {
        if (population < maxPopulation) {
            inhabitants.add(i);
            population++;
            return true;
        }
        return false;
    }
    
    public int calculateDefenseScore() {
        defenseScore = 0;
        for (Building b : buildings) {
            defenseScore += b.getLevel() * 10;
        }
        return defenseScore;
    }
    
    public boolean canBuild(Building b) {
        return area + 10 <= maxArea;
    }
}
\end{lstlisting}
\end{spacing}

\clearpage

\section{Core Package - Player Class}
\begin{spacing}{0.8}
\begin{lstlisting}[language=Java]
package core;

import buildings.Building;
import inhabitants.Inhabitant;
import engine.AttackResult;

public class Player {
    private String playerName;
    private String playerId;
    private Village village;
    private int rank;
    private int attackWins;
    private int attackLosses;
    private int defenseWins;
    private int defenseLosses;
    private int totalLoot;
    
    public Player(String name) {
        this.playerName = name;
        this.playerId = generatePlayerId();
        this.village = new Village(name + "'s Village");
        this.rank = 0;
    }
    
    private String generatePlayerId() {
        return "P" + System.currentTimeMillis();
    }
    
    public boolean buildBuilding(Building b) {
        return village.addBuilding(b);
    }
    
    public boolean upgradeBuilding(Building b) {
        b.upgrade();
        return true;
    }
    
    public boolean upgradeInhabitant(Inhabitant i) {
        i.upgrade();
        return true;
    }
    
    public void updateRank() {
        rank = (attackWins * 10) - (attackLosses * 5) 
            + (defenseWins * 8) - (defenseLosses * 3);
    }
    
    public Village getVillage() {
        return village;
    }
}
\end{lstlisting}
\end{spacing}

\clearpage

\section{Engine Package - GameEngine Class}
\begin{spacing}{0.8}
\begin{lstlisting}[language=Java]
package engine;

import java.util.List;
import java.util.ArrayList;
import core.Player;
import core.Village;
import buildings.Building;

public class GameEngine {
    private long currentTime;
    private double gameSpeed;
    private AttackSimulator attackSimulator;
    private VillageGenerator villageGenerator;
    private ScoreCalculator scoreCalculator;
    private List<Player> players;
    
    public GameEngine() {
        this.currentTime = System.currentTimeMillis();
        this.gameSpeed = 1.0;
        this.attackSimulator = new AttackSimulator();
        this.villageGenerator = new VillageGenerator();
        this.scoreCalculator = new ScoreCalculator();
        this.players = new ArrayList<>();
    }
    
    public void startGame() {
        currentTime = System.currentTimeMillis();
    }
    
    public void updateTime() {
        currentTime = System.currentTimeMillis();
    }
    
    public void processUpgrades(Village v) {
        // Process building and inhabitant upgrades
    }
    
    public boolean allowBuild(Village v, Building b) {
        return v.canBuild(b);
    }
    
    public void addPlayer(Player p) {
        players.add(p);
    }
}
\end{lstlisting}
\end{spacing}

\section{Engine Package - AttackResult Class}
\begin{spacing}{0.8}
\begin{lstlisting}[language=Java]
package engine;

import resources.Cost;

public class AttackResult {
    private boolean success;
    private double successRate;
    private Cost loot;
    private int attackerLosses;
    private int defenderLosses;
    private int stars;
    
    public AttackResult(boolean success, Cost loot) {
        this.success = success;
        this.loot = loot;
        this.attackerLosses = 0;
        this.defenderLosses = 0;
        this.stars = success ? 3 : 0;
    }
    
    public boolean isSuccess() { return success; }
    public Cost getLoot() { return loot; }
    public int getStars() { return stars; }
}
\end{lstlisting}
\end{spacing}

\clearpage

\section{UI Package - GameUI Class}
\begin{spacing}{0.8}
\begin{lstlisting}[language=Java]
package ui;

import core.Player;
import engine.GameEngine;

public class GameUI {
    private Player currentPlayer;
    private GameEngine gameEngine;
    
    public GameUI(Player player, GameEngine engine) {
        this.currentPlayer = player;
        this.gameEngine = engine;
    }
    
    public void displayVillage() {
        System.out.println("=== Village ===");
        // Display village information
    }
    
    public void displayBuildMenu() {
        System.out.println("=== Build Menu ===");
        System.out.println("1. Farm");
        System.out.println("2. Gold Mine");
        System.out.println("3. Archer Tower");
    }
    
    public void displayAttackMenu() {
        System.out.println("=== Attack Menu ===");
        // Display attack options
    }
    
    public void handleUserInput() {
        // Handle user input for game actions
    }
}
\end{lstlisting}
\end{spacing}

\end{document}
